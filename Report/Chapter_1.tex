\chapter{Introduction}
\thispagestyle{empty}
Symmetry is mostly known to be an observational notion in daily life. People by far more observe it  or use it in an aesthetic manner rather than put it into use. Thats simply because the cause and effect relationship between symmetry and engineering is still vague and the tools of trade when an engineer advances in her design is inadequate fro the symmetry perspective. 

Despite of its lack in the world of engineering, the world of physics is using symmetries for more than a decade, to understand the laws of nature. Assuming that there \textbf{are} laws of nature, in order to investigate them, symmetry of a phenomenon in nature is one of the first to check. Though the world of science tries to discover the inner workings of nature, by making measurements and by linking phenomena, they have little idea what the \textbf{nature} of these fundamental laws that one unfolds those little by little. At first, metaphysical nature of the law of the nature was popular amongst the philosophers, but it is hardly mentioned with notable credit since the mid 19th century. \cite{fraassen_laws_1990} The new way to look at the fundamental laws changed in form when people realized that the symmetry language, with the help of the group theory, was very helpful in explaining the then-unexplored areas of particle physics.  Symmetry groups of Lie type were so successful in interpreting the results of the experiments, symmetry in the space-time as we know it started to be taken for granted. For every different symmetry group that was proposed, a set of particles were generated on paper, then spins, colors and other fundamental properties were investigated. As the chart of subatomic particles populated in the mid 1900's, the symmetry groups started to dictate how many of them must there be out there, or which families these particles must be present in. Right now, the language of sub-atomic physics is dominated with the language of symmetry, since its ability to link the observation to the reality is still unsurpassed. \cite{penrose_road_2007}

There are other branches of science that use symmetry arguments in order to explain other macroscopic phenomena. One of them is crystallography, which is in close contact to electrical engineering via the semiconductor. The crystals in nature, their atomic arrangements are ideally applicable to symmetry groups acting on the three dimensional space. They exhibit spatial symmetries, some of them are rotational symmetries, and many of them are translational symmetries. When a perfect crystal is translated in space a length of its lattice constant, will remain unchanged. Assuming that it is large enough, this means that the problem that is to be solved is exactly the same of the previous one (the problem in the untranslated coordinates one had). Though other symmetries might have caused a different constraints, translational symmetry dictates the crystal to have periodic solutions to the Schr\"{o}dinger wave equation. This constraints affects the energies of the fermions (electrons in this case). Introduction of the symmetry to the problem affects the energy distribution of one electron under such circumstances,in such a way that some bands of energy turn out to be forbidden for an electron to propagate with. This is not the case for a free-electron, since its kinetic energy is a quadratic function of its momentum, and an electron can have any momentum within a broad range. The symmetry in Silicon for example, is the reason behind its well celebrated bandgap. This bandgap gives the engineers to have a control over the behavior of electrons, and this control is achieved by breaking the symmetry of the crystal for a desired amount. \cite{ashcroft_solid_1976,colclaser_materials_1986}

Though the language of symmetry is seldom spoken behind the walls of the room of crystallography, its applicability does not degrade over time. The set of tools acquired when one understands group theory and uses it in order to understand symmetries of systems of real life, is totally valid for any symmetry that any type of system can have. This work will investigate symmetries of dynamical system networks, networks coupled to themselves in a symmetric manner, and will try to shed light on how symmetry in capable of transforming an arbitrary problem to a problem with constraints.

If engineering is the act of using mathematical and physical reality in order to create systems that have exhibit some sort of control, an approach to the problem with the knowledge of symmetries and their implications on such systems, one can use this set of tools to mold the system into ones liking.  For example, as the reader may see in the following chapters, symmetries can be used in order to dictate limits on phase portraits of nonlinear dynamical systems of high dimensions. This may have applications in nonlinear control theory, or theory of oscillators, bot more importantly than its areas of direct applicability, is the vision it introduces into observations an engineer would make.

The first chapter after this introduction, introduces group theory, to an extent that is required to follow and understand the flow on this work. The second chapter creates the link between the symmetry groups and the object of interest (which may be a different one depending on what is investigated) , namely, an arbitrary dynamical system network. Theorems are shown and explained in this chapter, that will be useful in other sections. In the fourth chapter, an example is given for a network of dynamical systems, and its properties is investigated from the perspective that is delineated. Also, the results are simulated with a dynamical system simulator named XPPAUT, and the results obtained from the symmetry perspective is backed up.
